\documentclass{article}
%\documentclass{aastex63}
\usepackage[utf8]{inputenc}
\usepackage[a4paper, total={6in, 10in}]{geometry}
\usepackage{indentfirst}
\usepackage{hyperref}
\usepackage{amsmath}

%-- Journal definitions for the bibliography
\def\latex{\LaTeX{}}
%\def\apss{Ap\&SS}             % Astrophysics and Space Science
\def\aap{Astron.\ Astrophys.}                % Astronomy and Astrophysics
\def\aaps{Astron.\ Astrophys.\ Suppl.\ Ser.}                % Astronomy and Astrophysics
%\def\aapr{A\&A~Rev.}          % Astronomy and Astrophysics Reviews
%\def\pasa{PASA}	% Publications of the Astronomical Society of Australia
%\def\aplett{ApL}	 % Astrophysics Letters
%\def\nar{NewAR}          % New astronomy reviews
%\def\physrep{Phys.~Rep} % Physics reports
%\def\prd{Phys.~Rev.~D} % Physics reports

\def\aap{Astron. \& Astrophys.}                % Astronomy and Astrophysics
\newcommand{\aj}{Astron. J.}
\newcommand{\apjs}{The Astrophysical Journal}
\newcommand{\apj}{The Astrophysical Journal}
\newcommand{\apjl}{ApJ Lett.}
\newcommand{\mnras}{Mon. Not. R. Astro. Soc.}
\newcommand{\pasa}{Publ. Astron. Soc. Australia}
\newcommand{\nat}{Nature}
\newcommand{\sci}{Science}
\newcommand{\jcap}{J.Cosmol.Astropart.Phys.}
\newcommand{\procspie}{Proc. SPIE.}
\newcommand{\pasp}{Pub. Astron. Soc. Pacific}
\newcommand{\annrevastro}{Annu. Rev. Astron. Astrophys.}
\newcommand{\araa}{Annu. Rev. Astron. Astrophys.}

\title{Notes on GR 0 \\ Review of SR \& Introduction to vector and tensor algebra}
\author{Hyerin Cho}
%\date{\today}
\date{\vspace{-5ex}}

\usepackage{amsthm}

\theoremstyle{plain}
\newtheorem{thm}{Theorem}[section] % reset theorem numbering for each section

\theoremstyle{definition}
\newtheorem{defn}[thm]{Definition} % definition numbers are dependent on theorem numbers
\newtheorem{exmp}[thm]{Example} % same for example numbers


\usepackage{natbib}
\usepackage{graphicx}

%% Enable block commenting
\usepackage[]{verbatim}

%% Enable textcolor
\usepackage{color}

\usepackage{tensor}
\usepackage{braket}

\graphicspath{{./}{figures/}}

\begin{document}

\maketitle

\section{Special Relativity} \label{sec:sr}

\subsection{Invariance of an interval}
\begin{eqnarray}
    \Delta s^2 & = & - (\Delta t)^2 + (\Delta x)^2\\
    \Delta s^2 & = & \Delta \bar{s}^2  
\end{eqnarray}

\begin{defn} Proper time \end{defn} \label{eq:tau}
\begin{equation}
    d\tau^2 = -d\vec{x} \cdot d\vec{x} = -ds^2
\end{equation}

\subsection{Lorentz transformation}
\begin{itemize}
  \item $\bar{t} = \frac{t}{\sqrt{(1-v^2)}} - \frac{v x}{\sqrt{(1-v^2)}}$
  \item $\bar{x} = \frac{-v t}{\sqrt{(1-v^2)}} + \frac{x}{\sqrt{(1-v^2)}}$
  \item $\bar{y} = y$
  \item $\bar{z} = z$
\end{itemize}

\subsection{Velocity composition law}

\begin{equation}
    W = \frac{\Delta x}{\Delta t} = \frac{\bar{W}+v}{1+\bar{W}v}
\end{equation}
where $\bar{W} = \Delta \bar{x} / \Delta \bar{t}$. Classical limit goes to Galilean law of velocity addition $W = \bar{W} + v$.

\section{Vector Algebra} \label{sec:va}

\subsection{Basics}
\begin{enumerate}
    \item Definition of a vector $\Delta \vec{x} \rightarrow \Delta x^\alpha$ ($\alpha,\beta, \dots = 0,1,2,3$ and $i, j, \dots = 1,2,3$)
    \item Einstein summation convention $A_\alpha B^{\alpha} = \Sigma_{\alpha=0}^n A_\alpha B^\alpha$
    \item Basis vectors $\vec{e_\alpha}$: $\vec{A} = A^\alpha \vec{e_\alpha}$ \\
    Basis vectors transform like: $\vec{e_\alpha} = \Lambda\indices{^{\bar{\beta}}_\alpha} \vec{e\indices{_{\bar{\beta}}}}$
    \item Lorentz transformation matrix $\Delta x^{\bar{\alpha}} = \Lambda\indices{^{\bar{\alpha}}_\beta} \Delta x^{\beta}$\\
    \begin{eqnarray}
    [\Lambda\indices{^{\bar{\alpha}}_\beta}] =
     \left[
    \begin{array}{cccc}
    \gamma & -v \gamma & 0 & 0 \\
    -v \gamma & \gamma & 0 & 0 \\
    0 & 0 & 1 & 0 \\
    0 & 0 & 0 & 1
    \end{array}     \right]\nonumber
    \end{eqnarray}
    \item Inverse transformation for $\Lambda\indices{^{\bar{\alpha}}_\beta} = \Lambda\indices{^{\bar{\alpha}}_\beta}(\vec{v})$. 
    \begin{eqnarray}
        \Lambda\indices{^{\bar{\alpha}}_\beta} (-\vec{v}) & = & \Lambda\indices{_\beta^{\bar{\alpha}}}\\
        \Lambda\indices{_{\bar{\beta}}^\nu} \Lambda\indices{^{\bar{\beta}}_\alpha} & = & \delta^\nu_\alpha
    \end{eqnarray}

\end{enumerate}

\subsection{Four-velocity and four-momentum}

\begin{defn} \textbf{Momentarily comoving reference frame (MCRF)} is an inertial frame which momentarily has the same velocity as the particle. \end{defn} 

\begin{defn} \textbf{Four-velocity $\vec{U}$} of an accelerated particle is $\vec{e_0}$ basis vector of its MCRF at that event. \end{defn}

\begin{defn} \textbf{Four-momentum} $\vec{p} = m \vec{U}$ \end{defn}

\noindent\textit{(example)} Components of $\vec{U}$ and $\vec{p}$ for a particle of rest mass $m$ moving with velocity $\mathbf{v}$ in the $x$ direction.
\begin{eqnarray}\nonumber
    U^0 & = & (1-v^2)^{-1/2},\\\nonumber
    U^1 & = & v(1-v^2)^{-1/2},\\\nonumber 
    U^2 & = & 0,\\\nonumber
    U^3 & = & 0
\end{eqnarray}
notice it is natural to set $E := p^0$.\\

\subsection{Scalar product}
\begin{defn} \textbf{Scalar product of two vectors} \end{defn}
\begin{equation}
    \vec{A}\cdot\vec{B} = -A^0B^0+A^1B^1+A^2B^2+A^3B^3
\end{equation}

\begin{defn} \textbf{Orthonormal} \end{defn}
\begin{equation}
    \vec{e_\alpha} \cdot \vec{e_\beta} = \eta_{\alpha\beta},
\end{equation}
where $\eta_{\alpha\beta}$ is a generalized Kronecker delta (actually a metric).

\subsection{Rather useful expressions}
\begin{eqnarray}
    \vec{U} & = & \frac{d\vec{x}}{d\tau},\\
    \vec{a} & = & \frac{d\vec{U}}{d\tau}, (\vec{U}\cdot\vec{a} = 0)\\
    E^2 & = & m^2 + \Sigma_i (p^i)^2
\end{eqnarray}

\subsection{Photons}
Since $d\vec{x} \cdot d\vec{x} = 0$, $m^2 = -\vec{p}\cdot\vec{p} = 0$. Also notice that a photon has \textit{no} rest frame.

\section{Tensor Algebra} \label{sec:ta}

\subsection{Metric tensor components $\eta_{\alpha\beta}$}

\begin{eqnarray}
    \vec{A} \cdot \vec{B} & = & A^\alpha B^\beta (\vec{e_\alpha} \cdot \vec{e_\beta})\nonumber\\
    & = & A^\alpha B^\beta \eta_{\alpha\beta}\nonumber
\end{eqnarray}
%This is \textit{frame-invariant} way of writing $-A^0B^0+A^1B^1+A^2B^2+A^3B^3$

\subsection{Definition of tensors}

This part can be as mathematical as you would like, but I avoid the mathematical definitions for this study group. Please read to Ch.3 of \citet{1985schutz} since it is worth checking out mathematical views at some point during your studies.
\begin{comment}
\begin{defn} \textbf{Tensor of type $\binom{0}{N}$} is a function of $N$ vectors into the real numbers, which is linear in each of its $N$ arguments. \end{defn}

\noindent \textit{(example)} \textbf{metric tensor} $\mathbf{g}$
\begin{equation}
    \mathbf{g}(\vec{A},\vec{B}) := \vec{A} \cdot \vec{B}
\end{equation}

\begin{defn} \textbf{Components of a tensor} of type $\binom{0}{N}$ are the values of the function when its arguments are the basis vectors ${\vec{e_\alpha}}$ of the frame. $\mathbf{g}(\vec{e_\alpha},\vec{e_\beta})$. \end{defn}
\end{comment}

But for now, it is enough to accept that a metric is one example of a tensor.

\subsection{Contravariant vectors, Covariant vectors, Tensors}

\begin{defn} \textbf{Contravariant vector} = vectors $p^\alpha$ \end{defn}

\begin{defn} \textbf{Covariant vector} = covector = one-forms $p_\alpha = \eta\indices{_\alpha_\beta} p^\beta$ \end{defn}
\noindent \textit{(example)} Gradiant is a one-form. $\frac{\partial \phi}{\partial x^\alpha} := \phi_{, \alpha}$, and $\frac{\partial}{\partial x^\alpha} := \partial_{\alpha}$. Then, we see that $x^\alpha_{,\beta} = \delta\indices{^\alpha_\beta}$

\begin{defn} \textbf{contraction} $p^\alpha p_\beta = p_\alpha p^\beta$ \end{defn}

A simple way to think of tensors $T\indices{^{\alpha\beta\cdots}_{\gamma\delta\cdots}}$ is that they are outer products of one-forms and vectors. Index raising and lowering can be done with the help of a metric tensor $\eta_{\mu\nu}$ like the following:
\begin{equation}
    T\indices{^{\alpha\beta}_\gamma} = \eta\indices{_{\gamma\mu}} T\indices{^{\alpha\beta\mu}}
\end{equation}

\noindent \textit{(excercise)} $\eta\indices{^\alpha_\beta}$?

\subsection{Differentiation of tensors}
\begin{equation}
    d T\indices{^\alpha_\beta}/ d\tau = T\indices{^\alpha_{\beta,\gamma}} U^\gamma
\end{equation}

\bibliographystyle{aasjournal}
\bibliography{references}
\end{document}